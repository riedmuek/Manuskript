\section{Methodologischer Rahmen}
Der methodologische Rahmen dieser Arbeit basiert auf den Konzept der Gesellschaftsgeschichte nach Jürgen Kocka. % TODO Quelle ergänzen
Hiernach werden bestehende geschichtswissenschaftliche Methoden um soziologische Theorien und Methoden ergänzt.
Dieses Vorgehen ist erforderlich, um die Erforschung der ideologischen Einflüsse auf die Kriminalpolizei der DDR nicht nur als isolierte und für sich rational handelte Institution zu ermöglichen, sondern auch den Blick auf weitere Ebenen zu weiten, die die Kriminalpolizei in die gesellschaftliche Strukturen und Prozesse einbettete. 
Hierdurch wird der Fokus von einer reinen Ereignisgeschichte auf die Strukturen, die Prozesse und die Akteure -- die Personen aber auch anderen Institutionen sein können -- verschoben. 
Nach der Vorstellung Kockas soll durch das Konzept der Gesellschaftsgeschichte eine \qq{geschärfte Analyse} mit einem ausgeprägten Erklärungswissen % TODO Was ist damit gemeint?
verbunden werden. % TODO QUELLE ergänzen 
Diese Herangehensweise ist besonders geeignet für die Untersuchung der ideologischen Einflüsse auf die Arbeit und Mitglieder der Kriminalpolizei der DDR, da sie die Wechselwirkungen zwischen verschiedenen gesellschaftlichen Bereichen und Ebenen berücksichtigt.
Das ist insofern erforderlich, da wie in \cref{sec:Ideologie} dargestellt, Ideologien auf der individuellen sowie interpersonellen Ebene mit gesellschaftlichen Aushandlungsprozessen und Herrschaftspraktiken\autocite[siehe hierzu][21-23]{Lindenberger1999} % TODO Quelle / Referenz zu Lindenberger? 
verbunden sind. % TODO Aspekt im Kapitel zur Ideologie aufnehmen
\par Das Konzept Kockas wird in dieser Arbeit um das Konzept des \qq{Eigen-Sinns}\autocite{Lindenberger1999} nach Thomas Lindenberger ergänzt.
Dieser Ansatz ermöglicht die individuellen Handlungsspielräume und Bewertungen der Akteuere differenzierter zu betrachten, da Lindenberger davon ausgeht, dass die Individuen selbst in stark reglementieren Systemen, wie z.\,B. die DDR, über eine Teilautonomie verfügen, die in Handlungen enden, die von dem jeweiligen Herrschaftssystem nicht in dieser Form so vorgesehen waren.\autocite[24]{Lindenberger1999}
Durch die Integration des Eigen-Sinn-Konzepts in die Methodik dieser Arbeit wird es möglich:
\begin{itemize}
  \item Die Unterschiede zwischen den offiziellen Vorgaben und der tatsächlichen Praxis der Kriminalpolizei zu untersuchen.
  \item Die Strategien der Anpassung und der Umdeutung innerhalb des institutionellen Rahmen zu identifizieren.
  \item Die Beziehungen zwischen den einzelnen Akteuren, wie die Gesellschaft der DDR, die kooperierende (staatlichen) Institutionen, den Personengruppen innerhalb der Kriminalpolizei und Individuen selbst besser zu erfassen.
\end{itemize}
Diese methodische Erweiterung erlaubt es, nicht nur die strukturellen und ideologischen Aspekte der Kriminalpolizei zu analysieren, sondern auch die subjektiven Erfahrungen und Handlungsweisen der beteiligten Akteure zu berücksichtigen. 
Dieser Ansatz ermöglicht eine differenziertere Darstellung des Spannungsfelds zwischen ideologischer Prägung und individuellen und institutionellen Handelns.\par

\par Um sich den Untersuchungsgegenstand umfänglich zu betrachten, wurden die auszuwertenden Daten im Rahmen einer Datentriangulation aus verschiedenen Quellen gewonnen. 
Unter Datentriangulation wird der Einsatz von mindestens zwei oder mehreren Datenquellen verstanden, um ein und dasselbe Phänomen zu untersuchen. % TODO Quellen ergänzen
Dieser Ansatz erlaubt es, die Forschungsergebnisse zu vergleichen, die Stärken und Schwächen der einzelnen Methoden und Daten zu erkennen und die gewonnenen Erkenntnisse schließlich zu einem gemeinsamen Bild zusammenzusetzen.\autocites[13]{Flick2011}[287]{Hussy2013}
Zur Analyse der vielfältig erhobenen Daten wurden neben den historischen Methoden wie z.\,B. die Quellenkritik weitere Methoden aus der qualitativen Sozialforschung herangezogen. 
Die umfassen die inhaltliche Auswertung von periodisch erschienene Zeitschriften der Deutschen Volkspolizei, Abschlussarbeiten aus den Fachschulen der Volkspolizei, DDR-Fachbüchern -- insbesondere im Fachbereich Kriminalistik und Kriminologie -- und im Rahmen dieser Studie geführten Interviews von ehemaligen Kriminalpolizisten der DDR als Zeitzeugen.
Zudem wurden systematisch Weisungen und Gesetze ausgewertet, die die Kriminalpolizei betrafen. 
In einem weiteren Schritt wurden Akten der \acrshort{bstu} und der Humboldt-Universität zu Berlin erhoben und in die inhaltliche Auswertung mit einbezogen.
\par Im Folgenden werden die spezifischen methodischen Ansätze und Forschungsschwerpunkte erläutert, die für diese Untersuchung relevant waren. 
Dabei wird besonderes Augenmerk auf die Anwendung sozialwissenschaftlicher Konzepte auf historische Fragestellungen gelegt, um ein tiefergehende Verständnis der Rolle und Funktion der Kriminalpolizei im Kontext des Herrschaftssystem innerhalb der DDR zu entwickeln.

\section{Quellenkritik}
Der Datenkorpus besteht aus zwei Arten von Quellen, die sich bezüglich des Zeitpunkts ihrer Entstehung unterscheiden.
Ein Teil der Daten wurde während der Studie erstellt:
In dieser Gruppe sind die Interviews der Zeitzeugen einzuordnen.
Die zweite Gruppe der Daten entstand im Vergleich zu den Interviews bereits im untersuchten Zeitraum, also während der Existenz der DDR.
Hierbei handelt es sich um Geschichtsquellen bzw. historische Quellen, da es sich hierbei um \qq{Texte, Gegenstände oder Tatsachen [handelt], aus denen Kenntnis der Vergangenheit gewonnen werden kann}\autocite[29]{Kirn1959}.
Aus diesem Umstand, dass es sich um historische Quellen handelt, ergibt sich die Notwendigkeit einer Quellenkritik.\autocite[78]{Eckert2019} 
Hierbei wird der \qq{besonderen Umgang mit überlieferten Gegenständen, durch den diese erst historisch-wissenschaftlichen Erkenntniswert erhalten}\autocite[45]{Jordan2018} verstanden.
Dieser besondere Umgang umfasst die Analyse verschiedener Facetten der Quelle selbst.\footnote{Siehe hierzu die Fragestellungen zu Kritik und Interpretation nach \textcite[162]{Borowsky1989}.}
Das grundsätzliche Anliegen ist die Beschreibung der Quelle anhand deren Merkmale und die Prüfung, ob die Quellen das sind und ausgeben, wofür sie sich ausgeben. % TODO Quelle Bernheim 1907: 114 
Im Rahmen dieser Arbeit muss ein besonderer Augenmerk darauf gelegt werden, ob die Entstehung der Quellen verzerrenden Faktoren unterlagen.
In einem repressiven System wie die DDR sind insbesondere die Fragen zu Ersteller, Adressat und Zweck der Entstehung der Quelle von besonderen Interesse: 
Vor allem ist zu klären, wer die Urheber der Quelle sind und welche Motive sich erkennen lassen, warum diese Quelle entstanden ist.
Dabei spielen besondere Begriffe und Themen eine wichtige Rolle.
Methodisch werden die Quellen einer äußeren Quellenkritik (Echtheit und Vollständigkeit der Quelle)\autocite[67]{Budde2008} und einer inneren Quellenkritik (Perspektive, Wertungung und Widersprüche im Inhalt)\autocite[67]{Budde2008} unterzogen.\footnote{Vergleiche hierzu die umfangreiche Analysefragen im Rahmen der Quellenkrtik nach \textcite[78\psq]{Eckert2019} und \textcite[67]{Budde2008} «.} 

%
%Citations:
%[1] https://www.semanticscholar.org/paper/46eabbdbf0a7a9ae6f808345159ff70fb0c7ed32
%[2] https://www.semanticscholar.org/paper/7da644222060e8c8cee486b014236e2861c89630
%[3] https://www.semanticscholar.org/paper/7068fd41a5ef0fb041ecf34cf4fbeeff292c8112
%[4] https://www.semanticscholar.org/paper/bbc0b92e91e6a37b67390bd57bbb22729d8ee599
%[5] https://www.semanticscholar.org/paper/48003519783adea5383a7ad5854c8709db25bdaa
%[6] https://www.semanticscholar.org/paper/cf39e86868eddf0394489440f5cd24370f38d975
%[7] https://www.semanticscholar.org/paper/fdf26d8ee97dc4ad8b51cc0892fa684e7f2b000f