\section{Methodologischer Rahmen}
Der methodologische Rahmen dieser Arbeit basiert auf den Konzept der Gesellschaftsgeschichte nach Jürgen Kocka. % TODO Quelle ergänzen
Hiernach werden bestehende geschichtswissenschaftliche Methoden um soziologische Theorien und Methoden ergänzt.
Dieses Vorgehen ist erforderlich, um die Erforschung der ideologischen Einflüsse auf die Kriminalpolizei der DDR nicht nur als isolierte und für sich rational handelte Institution zu ermöglichen, sondern auch den Blick auf weitere Ebenen zu weiten, die die Kriminalpolizei in die gesellschaftliche Strukturen und Prozesse einbettete. 
Hierdurch wird der Fokus von einer reinen Ereignisgeschichte auf die Strukturen, die Prozesse und die Akteure -- die Personen aber auch anderen Institutionen sein können -- verschoben. 
Nach der Vorstellung Kockas soll durch das Konzept der Gesellschaftsgeschichte eine \qq{geschärfte Analyse} mit einem ausgeprägten Erklärungswissen % TODO Was ist damit gemeint?
verbunden werden. % TODO QUELLE ergänzen 
Diese Herangehensweise ist besonders geeignet für die Untersuchung der ideologischen Einflüsse auf die Arbeit und Mitglieder der Kriminalpolizei der DDR, da sie die Wechselwirkungen zwischen verschiedenen gesellschaftlichen Bereichen und Ebenen berücksichtigt.
Das ist insofern erforderlich, da wie in \cref{sec:Ideologie} dargestellt, Ideologien auf der individuellen sowie interpersonellen Ebene mit gesellschaftlichen Aushandlungsprozessen und Herrschaftspraktiken\autocite[siehe hierzu][21-23]{Lindenberger1999} % TODO Quelle / Referenz zu Lindenberger? 
verbunden sind. % TODO Aspekt im Kapitel zur Ideologie aufnehmen
\par Das Konzept Kockas wird in dieser Arbeit um das Konzept des \qq{Eigen-Sinns}\autocite{Lindenberger1999} nach Thomas Lindenberger ergänzt.
Dieser Ansatz ermöglicht die individuellen Handlungsspielräume und Bewertungen der Akteuere differenzierter zu betrachten, da Lindenberger davon ausgeht, dass die Individuen selbst in stark reglementieren Systemen, wie z.\,B. die \abk{ddr} , über eine Teilautonomie verfügen, die in Handlungen enden, die von dem jeweiligen Herrschaftssystem nicht in dieser Form so vorgesehen waren.\autocite[24]{Lindenberger1999}
Durch die Integration des Eigen-Sinn-Konzepts in die Methodik dieser Arbeit wird es möglich:
\begin{itemize}
  \item Die Unterschiede zwischen den offiziellen Vorgaben und der tatsächlichen Praxis der Kriminalpolizei zu untersuchen.
  \item Die Strategien der Anpassung und der Umdeutung innerhalb des institutionellen Rahmen zu identifizieren.
  \item Die Beziehungen zwischen den einzelnen Akteuren, wie die Gesellschaft der DDR, die kooperierende (staatlichen) Institutionen, den Personengruppen innerhalb der Kriminalpolizei und Individuen selbst besser zu erfassen.
\end{itemize}
Diese methodische Erweiterung erlaubt es, nicht nur die strukturellen und ideologischen Aspekte der Kriminalpolizei zu analysieren, sondern auch die subjektiven Erfahrungen und Handlungsweisen der beteiligten Akteure zu berücksichtigen. 
Dieser Ansatz ermöglicht eine differenziertere Darstellung des Spannungsfelds zwischen ideologischer Prägung und individuellen und institutionellen Handelns.\par

\par Um sich den Untersuchungsgegenstand umfänglich zu betrachten, wurden die auszuwertenden Daten im Rahmen einer Datentriangulation aus verschiedenen Quellen gewonnen. 
Unter Datentriangulation wird der Einsatz von mindestens zwei oder mehreren Datenquellen verstanden, um ein und dasselbe Phänomen zu untersuchen. % TODO Quellen ergänzen
Dieser Ansatz erlaubt es, die Forschungsergebnisse zu vergleichen, die Stärken und Schwächen der einzelnen Methoden und Daten zu erkennen und die gewonnenen Erkenntnisse schließlich zu einem gemeinsamen Bild zusammenzusetzen.\autocites[13]{Flick2011}[287]{Hussy2013}
Zur Analyse der vielfältig erhobenen Daten wurden neben den historischen Methoden wie z.\,B. die Quellenkritik weitere Methoden aus der qualitativen Sozialforschung herangezogen. 
Die umfassen die inhaltliche Auswertung von periodisch erschienene Zeitschriften der Deutschen Volkspolizei, Abschlussarbeiten aus den Fachschulen der Volkspolizei, DDR-Fachbüchern -- insbesondere im Fachbereich Kriminalistik und Kriminologie -- und im Rahmen dieser Studie geführten Interviews von ehemaligen Kriminalpolizisten der DDR als Zeitzeugen.
Zudem wurden systematisch Weisungen und Gesetze ausgewertet, die die Kriminalpolizei betrafen. 
In einem weiteren Schritt wurden Akten der \acrshort{bstu} und der Humboldt-Universität zu Berlin erhoben und in die inhaltliche Auswertung mit einbezogen.
\par Im Folgenden werden die spezifischen methodischen Ansätze und Forschungsschwerpunkte erläutert, die für diese Untersuchung relevant waren. 
Dabei wird besonderes Augenmerk auf die Anwendung sozialwissenschaftlicher Konzepte auf historische Fragestellungen gelegt, um ein tiefergehende Verständnis der Rolle und Funktion der Kriminalpolizei im Kontext des Herrschaftssystem innerhalb der DDR zu entwickeln.

\section{Quellenkritik}
Der Datenkorpus besteht aus zwei Arten von Quellen, die sich bezüglich des Zeitpunkts ihrer Entstehung unterscheiden.
Ein Teil der Daten wurde während der Studie erstellt:
In dieser Gruppe sind die Interviews der Zeitzeugen einzuordnen.
Die zweite Gruppe der Daten entstand zu den Interviews bereits im untersuchten Zeitraum, also während der Existenz der DDR, und wurde lediglich nur in das Set der zu untersuchenden Daten aufgenommen.
Hierbei handelt es sich um Geschichtsquellen bzw. historische Quellen, da es sich hierbei um \qq{Texte, Gegenstände oder Tatsachen [handelt], aus denen Kenntnis der Vergangenheit gewonnen werden kann}\autocite[29]{Kirn1959}.
Aus dem Umstand, dass es sich um historische Quellen handelt, ergibt sich die Notwendigkeit einer Quellenkritik.\autocite[78]{Eckert2019} 
Hierbei wird der \qq{besonderen Umgang mit überlieferten Gegenständen, durch den diese erst historisch-wissenschaftlichen Erkenntniswert erhalten}\autocite[45]{Jordan2018} verstanden.
Dieser besondere Umgang umfasst die Analyse verschiedener Facetten der Quelle selbst.\footnote{Siehe hierzu die Fragestellungen zu Kritik und Interpretation nach \textcite[162]{Borowsky1989}.}
Das grundsätzliche Anliegen ist die Beschreibung der Quelle anhand deren Merkmale und die Prüfung, ob die Quellen das sind und ausgeben, wofür sie gehalten werden.\footcite[114]{Bernheim1907}
Im Rahmen dieser Arbeit muss ein besonderer Augenmerk darauf gelegt werden, ob die Entstehung der Quellen verzerrenden Faktoren unterlagen.
In einem repressiven System wie die DDR sind insbesondere die Fragen zu Ersteller, Adressat und Zweck der Entstehung der Quelle von besonderen Interesse: 
Vor allem ist zu klären, wer die Urheber der Quelle sind und welche Motive sich erkennen lassen, warum diese Quelle entstanden ist.
Dabei spielen besondere Begriffe und Themen eine wichtige Rolle.
Methodisch werden die Quellen einer äußeren Quellenkritik (Echtheit und Vollständigkeit der Quelle)\autocite[67]{Budde2008} und einer inneren Quellenkritik (Perspektive, Wertungung und Widersprüche im Inhalt)\autocite[67]{Budde2008} unterzogen.\footnote{Vergleiche hierzu die umfangreiche Analysefragen im Rahmen der Quellenkrtik nach \textcite[78\psq]{Eckert2019} und \textcite[67]{Budde2008} .} 


\section{Oral History}
% >>> Überleitung ggf. später anpassen, wenn Kapitel umgestellt
Die in dieses Projekt einbezogenen historischen Quellen wie z.\,B. die Zeitschriften \emph{Forum der Kriminalistik} oder Fachbücher wie z.\,B. \emph{Sozialistische Kriminologie}, die zur Zeitpunkt der Existenz der \abk{ddr} entstanden sind, richten sich an die Leser der damaligen Zeit. 
Sie unterlagen damit in ihrem Inhalt und ihre Ausgestaltung Einschränkungen.\footnote{Siehe hierzu \cref{sec::auswertung-fachzeitschriften} und \cref{sec::auswertung-fachliteratur}.} 
Zudem erlauben diese Quellen nur wenige Einblicke in die informellen Strukturen der Deutschen Volkspolizei und die persönlichen Motivationslagen und Deutungsmuster der handelden Personen. % TODO Acronym Gen für DVP einfügen
Um diese Lücken zu schließen, wurde auf die Methode der Oral History zurückgegriffen.
% <<< Überleitung
\par 
\begin{quotation}
  \qq{Als Oral History bezeichnet man eine Methode der Geschichtswissenschaft, die auf die Untersuchung mündlicher Überlieferung historischer Inhalte gerichtet ist, bevorzugt mit Befragungen und Interviews arbeitet und naturgemäß fast ausschließlich in der Zeitgeschichte angewandt wird.}\autocite[162]{Jordan2018}
\end{quotation}
Ein wichtiger Vertreter dieser Methode ist Lutz Niethammer.
Dieses Instrument etablierte sich in Deutschland in den 1970er Jahren für historische Biografik und fokussierte  sich seitdem auf bisher von der Forschung vernachlässigte historische Subjekte.\autocite[168]{Budde2008a} 
Dabei wird von der Überlegung ausgegangen, dass das die Aussagen innerhalb der Interviews mehr Aussagekraft haben, als andere körperliche Überlieferungen, da hier das Geschilderte unmittelbar transportiert wird und diese Aussagen genauer sind, als sie auf andere Weise festgehalten werden können.
Deshalb sind hier Facetten aus den Ereignissen der Vergangenheit erfahrbar, die bisher noch nicht im Fokus standen.\autocite[351]{Wagner2021} % TODO Fremdzitat auflösen
Hierbei ist jedoch zu beachten, dass Oral History vor allem auf den Einzelnen abzielt, also auf die Mikrogeschichte\footnote{Mikrohistorie ist ein geschichtswissenschaftlicher Ansatz, der sich auf sehr kleine Forschungsgegenstände wie einzelne Gerichtsprozesse oder alltägliche Ereignisse konzentriert. Im Gegensatz zur Makroperspektive, die große geschichtliche Zusammenhänge betrachtet, versucht die Mikrohistorie durch die Analyse des Kleinen Erkenntnisse über das Große zu gewinnen. Sie unterscheidet sich dabei von Lokal- und Regionalgeschichte, die sich auf kleine geographische Räume fokussieren.\textcite[158]{Jordan2018}} .
Damit eignet sich dieser Ansatz, wie auch Thomas Lindenberger in seiner Habilitationsschrift\autocite[33]{Lindenberger2003} anmerkte, zur Rekonstruktion sozialer Interaktionen zwischen den einzelnen Akteuren. 
An dieser Perspektive fehlt es innerhalb der vorwiegend in schriftlicher Form vorliegenden Quellenlage. 
Der Einwand von Lindenberger, dass Oral History nur dann notwendig wäre, \qq{wenn keine schriftliche Überlieferung vorliegt bzw. keine, deren quellenkritische Lektüre die Rekonstruktion verschiedener Perspektiven und Standpunkte der beteiligten Akteure erlaubt}\autocite[33]{Lindenberger2003}, kann nicht gefolgt werden, da schriftliche Überlieferungen nur in der vorliegenden Form ausgewertet werden können. 
Im Interview kann -- solange es sich nicht um ein reines offenes, narratives Interview handelt -- durch Nachfragen Details herausgearbeitet werden, die in Schriftstücken aus den verschiedensten Gründen nicht enthalten sind.\par 
Darüber hinaus ermöglicht die Oral History eine stärkere Fokussierung auf subjektive Deutungsmuster wodurch nicht nur Fakten, sondern auch Motive offengelegt werden.
Durch die unmittelbare Interaktion zwischen Interviewer und Interviewtem können zudem wesentliche Kontextinformationen gewonnen werden, die bei der Erstellung von Dokumenten nicht festgehalten wurden, wie bspw. individuelle Einschätzungen zur Bedeutung bestimmter Ereignisse oder institutioneller Abläufe. 
Gerade im Zusammenhang mit den Strukturen und Praktiken der Deutschen Volkspolizei ist dies von Bedeutung, da verborgene Handlungsmotive und -abläufe, die nur selten Eingang in die schriftliche Dokumentation fanden, nur so ausgearbeitet werden können.\par 
Bei der Ausgestaltung der Art der Interviews mussten weitere Entscheidungen getroffen werden. 
Grundsätzlich sagt die Wahl von Oral History zunächst nur aus, dass Interviews innerhalb der Geschichtsforschung geführt werden sollen.
Hier stehen den Forscher verschiedene Arten der Interviewführung offen.
Diese umfassen u.\,a. strukturierte, Leitfaden-orientierte oder narrative Interviews.
Am geeignetsten für dieses Forschungsvorhaben waren die Leitfaden-orientierten Interviews. 




Zusammenfassend lässt sich festhalten, dass die Oral History im Kontext dieser Untersuchung als notwendige Perspektive zu den schriftlichen Quellen betrachtet werden muss. 
Zum einen füllt sie die Lücken, die durch fehlende oder unvollständige Darstellung innerhalb dieser Quellen entstanden sind, zum anderen rückt sie die subjektiven Perspektiven und Deutungsmuster der beteiligten Akteure mit in den Fokus. 


%\subsection{Beschreibung der Quellen}
%\subsection{Vorgehen und Umsetzung}
%\subsection{Kritik der Methode}

\section{Auswertung von Fachzeitschriften der \abk{dvp}}\label{sec::auswertung-fachzeitschriften}
\section{Auswertung von Fachliteratur der \abk{ddr}}\label{sec::auswertung-fachliteratur}
\section{Hermeneutik}
% Ausführung, welchen Stellenwert die Hermeneutik -- aufgrund der dünnen Forschungslage -- in diesem Forschungsvorhaben hat.
Ein wesentlicher Faktor bei Erschließen der Inhalte der Quellen stellt das Vorwissen dar.
Wie in \cref{sec:Stand-der-Forschung} darstellt, liegt nach mehr als 35 Jahren Mauerfall immer noch eine dürftige Quellenlage über das gesamte Sicherheitssystem -- insbesondere zur \abk{dvp} -- der \abk{ddr} vor.




%
%Citations:
%[1] https://www.semanticscholar.org/paper/46eabbdbf0a7a9ae6f808345159ff70fb0c7ed32
%[2] https://www.semanticscholar.org/paper/7da644222060e8c8cee486b014236e2861c89630
%[3] https://www.semanticscholar.org/paper/7068fd41a5ef0fb041ecf34cf4fbeeff292c8112
%[4] https://www.semanticscholar.org/paper/bbc0b92e91e6a37b67390bd57bbb22729d8ee599
%[5] https://www.semanticscholar.org/paper/48003519783adea5383a7ad5854c8709db25bdaa
%[6] https://www.semanticscholar.org/paper/cf39e86868eddf0394489440f5cd24370f38d975
%[7] https://www.semanticscholar.org/paper/fdf26d8ee97dc4ad8b51cc0892fa684e7f2b000f