\documentclass[
  fontsize=12pt,       % Schriftgröße
  paper=a4,            % Papierformat
  twoside=true,        % Zweispaltiges Layout für beidseitigen Druck
  openright,           % Kapitel beginnen auf der rechten Seite
  ngerman,             % Neue deutsche Rechtschreibung
  parskip=half,        % Absatzabstand
  headsepline%,         % Linie unter der Kopfzeile
  %footsepline          % Linie über der Fußzeile
]{scrbook}

\usepackage[T1]{fontenc}
\usepackage[utf8]{inputenc}
\usepackage[ngerman]{babel}
\RequirePackage[ngerman=ngerman-x-latest]{hyphsubst}

\usepackage[pdfborder={0 0 0}]{hyperref}
\usepackage{breakurl}

% >>> Zum Einstellen des Dokuments
	\usepackage{lipsum}
% <<< Zum Einstellen des Dokuments


% >>> noch zu prüfende Pakete und Einstellungen
	%\usepackage{framed}
	%\usepackage{pdfpages}
	%\usepackage{nameref}
% <<< noch zu prüfende Pakete und Einstellungen

% >>> Formatierung des Dokuments
	\usepackage[
		left=3cm,
		right=4cm,
		top=2cm,
		bottom=2cm,
		foot=1cm]
			{geometry}
	\usepackage{amssymb}
	\usepackage{lmodern}
	\usepackage{mathptmx}
	\usepackage[onehalfspacing]{setspace}
	%\usepackage[scaled]{helvet}
	%\renewcommand\familydefault{phv}
	\usepackage{pdflscape} % für Querformat
% <<< Formatierung des Dokuments



% >>> Eigene Formatierungen
	\newcommand{\qq}[1]{\glqq #1\grqq} % Text in Anführungszeichen
	\newcommand{\gq}[1]{\glq{}#1\grq{}}
% >>> Eigene Formatierungen

% >>> BibLaTeX-Zitation
	\usepackage[babel,german=guillemets]{csquotes}	% Anführungszeichen
%	\usepackage[%
%		style=authoryear-comp,%
%		isbn=false,%
%		doi=false,%
%		hyperref,%
%		dashed=false]{biblatex}
%	\bibliography{literatur}

	\usepackage[%
		style=verbose-note,%
		isbn=false,%
		doi=false,%
		hyperref,%
		dashed=false]{biblatex}
	\bibliography{literatur}


	%% sorgt dafür, dass bei Zitaten mit mehreren Autoren ein "\" im Text steht und im Literaturverzeichnis ein ";"
		\renewcommand*{\multinamedelim}{\addslash}
		\renewcommand*{\finalnamedelim}{\addslash}
		\AtBeginBibliography{%
			\renewcommand*{\multinamedelim}{\addslash}
			\renewcommand*{\finalnamedelim}{\addslash}
			\renewbibmacro*{name:andothers}{%
				\ifthenelse{\value{listcount}=\value{liststop}\AND
					\ifmorenames}
				{\space et.\,al.}
				{}}
			}
		\DeclareNameAlias{sortname}{last-first}
		\DefineBibliographyStrings{ngerman}{ 
		   andothers = {{et\,al\adddot}},             
		}

	\renewcommand*{\mkbibnamelast}[1]{\textsc{#1}}	    %Nachname des Autors in Kapitälchen
	\renewcommand*{\labelnamepunct}{\addcolon\addspace} % Doppelpunkt nach Name in BIB

	% Filter für Primär- und Sekundärliteratur
		\DeclareBibliographyCategory{primary}
		\DeclareBibliographyCategory{secondary}
		\defbibheading{primary}{\section*{Primärquellen}}
		\defbibheading{secondary}{\section*{Sekundärliteratur}}
% <<< BibLaTeX-Zitation

% >>> Abkürzungen

\usepackage[
    ngerman,                % Sprache auf Neu-Deutsch setzen
    toc,                    % Eintrag ins Inhaltsverzeichnis
    acronym,                % Unterstützung für Akronyme
    nonumberlist,           % Keine Seitenzahlen im Glossar
    style=long,             % Stil der Darstellung (verschiedene Optionen verfügbar)
    hyperfirst=false        % Erstes Vorkommen ohne Hyperlink
]{glossaries}

\makeglossaries

\newacronym{bstu}{BStU}{Der Bundesbeauftragte für die Unterlagen des Staatssicherheitsdienstes der ehemaligen Deutschen Demokratischen Republik}

% <<< Abkürzungen

% >>> Erstellen von Referenzen im Text
	\usepackage{hyperref}
	\usepackage{cleveref}
% <<< Erstellen von Referenzen im Text
